\documentclass[11pt,a4paper,german,notitlepage]{report}


%\usepackage[applemac]{inputenc} %Kommentierung entfernen,falls Mac
\usepackage[utf8]{inputenc} %Windows, Linux
%\usepackage[ansinew]{inputenc} %TeXnicCenter kann leider kein UTF-8

%Umlaute
%\usepackage[english]{babel}
\usepackage[ngerman]{babel}

%einige Variablen definieren
\newcommand{\titelthema}{Angriffssimulation und \\ strukturierte Datenerfassung}
\newcommand{\authorname}{Johannes Seitz}
\newcommand{\authormail}{johannes2.seitz@stud.uni-regensburg.de}
\newcommand{\matrikelnr}{2136257}
\newcommand{\abgabedatum}{24. Juli 2020}
\newcommand{\betreuer}{Marietheres Dietz \\Daniel Schlette}
\newcommand{\arbeitstyp}{Praxisseminar}
\newcommand{\Adresse}{Dechbettener Str. 1}
\newcommand{\PLZuOrt}{93049 Regensburg}



% Standardschriftart
%\usepackage{times}
%\usepackage{garamond}

% Default-Schriften (\rmdefault Times, 
%\sfdefault Helvetica und \ttdefault Courier
%besser als \usepackage{times} da auch Mathematikschriften 
%berücksichtigt werden
\usepackage{mathptmx}  %Times
\usepackage[scaled=.92]{helvet}
%\usepackage{mathpazo} %Palatino
%\usepackage[scaled=.96]{berasans}
\usepackage{courier}

%Grafiken
\usepackage{graphicx}
\usepackage{subfigure} 

% Grafiken in einem Unterordner speichern
\graphicspath{{./images/}}


% Tabellen
%\usepackage{tabularx}

% Seitenränder
\usepackage[a4paper]{geometry}
\geometry{top=26mm, bottom=19mm, left=50mm, right=25mm, includefoot}

% Seitenränder auf Deckblatt
\usepackage[strict]{changepage}

%Querseiten mit korrekter Kopf-/Fußzeile
\usepackage{pdflscape}


%sauberer Blocksatz, optischer Randausgleich
\usepackage{microtype}


%1,5facher Zeilenabstand nur im Fließtext
\usepackage{setspace}
\onehalfspacing

%einfaches Anpassen von Kopf- und Fußzeilen 
\usepackage{fancyhdr}

\rhead{\small\thepage}
\lhead{\footnotesize\leftmark}
\chead{}
\rfoot{\footnotesize{\authorname, \the\year}}
\lfoot{\footnotesize{\arbeitstyp}}
\cfoot{}


% zusätzliche Sonderzeichen
%\usepackage{textcomp} 


% Mathematisches
%\usepackage{amssymb}
%\usepackage{amsmath} 
%\usepackage{amsfonts} 


% Code und Pseudocode
%\usepackage[ruled]{algorithm}
%\usepackage{algpseudocode}
%\usepackage[boxed]{algorithm}
%\usepackage{algpseudocode}

%Silbentrennung
\usepackage[T1]{fontenc}



% PDF options
\usepackage[hyperfootnotes=false, pdfpagelabels]{hyperref}
\urlstyle{rm}
\hypersetup{%
bookmarksopen={true},
%bookmarksopenlevel={2},
bookmarksnumbered=true,
pdfstartpage={1},
pdftitle={\arbeitstyp},
pdfsubject={\titelthema},
pdfauthor={\authorname},
pdfkeywords={},
pdfcreator={hyperref},
pdfproducer={LaTeX with hyperref},
pdffitwindow={false},
pdfpagelayout={SinglePage}
}


\usepackage{url}

\usepackage{color}
\usepackage{colortbl} %Für Listings
  \definecolor{dunkelgrau}{rgb}{0.7,0.7,0.7}
  \definecolor{hellgrau}{rgb}{0.9,0.9,0.9}

\usepackage{listing}
  \renewcommand{\listlistingname}{Quelltextverzeichnis}

\usepackage{listings}
  \lstset{numbers=left, numberstyle=\tiny, basicstyle=\scriptsize, backgroundcolor=\color{hellgrau}}
  \lstset{captionpos=b, aboveskip=15pt, belowskip=8pt, showstringspaces=false}



%Abkürzungsverzeichnis, nur verwendete Abkürzungen darstellen
\usepackage[printonlyused]{acronym}

%Anhang
\usepackage{appendix}


%Bild-, Tabellenuntschriften schöner formatieren
\usepackage[format=hang,margin=10pt,font=small,labelfont=bf]{caption}

%PDF einbinden
%\usepackage{pdfpages}
%\includepdf[pages=1-4]{Meindoku.pdf}


%Euro 
%\usepackage{eurosym} 
% das Euro-Zeichen kann so eingefügt werden: \euro{}

%Lorem ipsum Auto-Text um Format zu zeigen
\usepackage{blindtext}


% Trennlinie unter der Kopfzeile
\renewcommand{\plainheadrulewidth}{0.4pt}


% Abstand zwischen Absätzen
%\parskip 3pt

% kein Erstzeileneinzug
%\setlength{\parindent}{0em}

\author{
\authorname\\
\texttt{\authormail}
}

\begin{document}
%
%%%%%%%%%%%%%%%%%%%%%%%%%%%%%%%%%%%%%%%%%%%%%%%%%%%%%%%%%	
%
\pagenumbering{roman}


\phantomsection
%\addcontentsline{toc}{chapter}{Deckblatt}
\pdfbookmark[0]{Deckblatt}{Deckblatt}
\label{Deckblatt}

% Eidesstattliche Erklärung notwendig



% Eidesstattliche Erklärung _nicht_ notwendig
\include{DeckblattPraxisseminar}



%Abstract einbinden
\phantomsection
%\addcontentsline{toc}{chapter}{Abstract}
\pdfbookmark[0]{Abstract}{abstract}
%
%%%%%%%%%%%%%%%%%%%%%%%%%%%%%%%%%%%%%%%%%%%%%%%%%%%%%%%%%%%
%
% Abstract
%

\thispagestyle{empty}

%Seite zentrieren
\begin{adjustwidth}{-2cm}{}

%Überschrift "Abstract" statt default "Zusammenfassung"
%\renewcommand{\abstractname}{Abstract}

%\begin{abstract}
%\label{abstract}
%Your abstract goes here...

%\blindtext

%\blindtext


\begin{Huge}\textbf{
\newline
\newline
\newline
\newline
\newline %Achtung: hier keine Leerzeile einfügen
Abstract}\end{Huge} \\ \\

\blindtext

\blindtext





%\end{abstract}

\end{adjustwidth}

%%%%%%%%%%%%%%%%%%%%%%%%%%%%%%%%%%%%%%%%%%%%%%%%%%%%%%%%%
%%% Ende Abstract


%%%%%%%%%%%%%%%%%%%%%%%%%%%%%%%%%%%%%%%%%%%%%%%%%%%%%%%%%


\newpage

% Keine Kopf-/Fußzeile auf den ersten Seiten
\pagestyle{fancyplain}

% "Kapitel" aus Kopfzeile weglassen
\renewcommand{\chaptermark}[1]{\markboth{\uppercase{\thechapter.\ #1}}{}}
% Nummerierungen nur bis Gliederungsebene 3 (\subsubsection)
\setcounter{secnumdepth}{3}
% Nur Überschriften bis Gliederungsebene 3 (\subsubsection) ins Inhaltsverzeichnis
\setcounter{tocdepth}{3}
\setcounter{page}{1}
\newpage


%Inhaltsverzeichnis
  \phantomsection
%  \addcontentsline{toc}{chapter}{Inhaltsverzeichnis}
  \pdfbookmark[0]{Inhaltsverzeichnis}{Inhaltsverzeichnis}
  \label{Inhaltsverzeichnis}
  \tableofcontents
  \newpage
  
%Abbildungsverzeichnis
  \clearpage
  \phantomsection
  \addcontentsline{toc}{chapter}{Abbildungsverzeichnis}
  \listoffigures
  \newpage

%Tabellenverzeichnis
  \clearpage
  \phantomsection
  \addcontentsline{toc}{chapter}{Tabellenverzeichnis}
  \listoftables
  \newpage

%Verzeichnis der Codelistings
  \clearpage
  \phantomsection
  \addcontentsline{toc}{chapter}{Listings}
  \lstlistoflistings
  \newpage

%Abkürzungsverzeichnis
% nur verwendete Abkürzungen werden dargestellt
  \clearpage
  \phantomsection
  \chapter*{Abkürzungsverzeichnis}
  \markboth{\uppercase{Abkürzungsverzeichnis}}{}
   \addcontentsline{toc}{chapter}{Abkürzungsverzeichnis}
   \begin{acronym}[MMMMMMMMM]
   %Kompaktansicht, Zeilenabstand verringern
   \setlength{\itemsep}{-\parsep}
   %Abkürzung fett
   \renewcommand*{\acsfont}[1]{{\textbf{#1}}}
   %auch für Abkürzungen Serifenschrift verwenden
   \renewcommand{\aclabelfont}[1]{\normalfont{\normalsize{#1}}\hfill}
      \acro{Bsp.}{Beispiel}
      \acro{SaaS}{Software as a Service}
      \acro{XML}{Extensible Markup Language}
   \end{acronym}
  \newpage

%Verhinderung von "Schusterjungen" und "Hurenkinder"
  \clubpenalty=10000
  \widowpenalty=10000 
  \displaywidowpenalty = 10000
  \tolerance=500 %Zeilenumbruch



%%%%%%%%%%%%%%%%%%%%%%%%%%%%%%%%%%%%%%%%%%%%%%%%%%%%%%%%%%%%%%%%%%%%%%%%%%

% Nummerierung wieder bei 1 beginnen
\setcounter{page}{1}

\pagestyle{fancy}
\pagenumbering{arabic}



%%%%%%%%%%%%%%%%%%%%%%%%%%%%%%% Einleitung %%%%%%%%%%%%%%%%%%%%%%%%%%%%%%%

%%%%%%%%%%%%%%%%%%%%%%%%%%%%%%%%% Einleitung %%%%%%%%%%%%%%%%%%%%%%%%%%%%%%%%%

\chapter{Einleitung}
\label{chap:Einleitung}


Wie sich in den Anfangsmonaten des Jahres bereits gezeigt hat, ist unser Wirtschaftssystem bei Weitem fragiler, als es viele Experten vorhersagen konnten. Innerhalb kürzester Zeit versetzte ein kleiner Virus die komplette Weltwirtschaft in einen ungewollten Ruhemodus.
Es zeigte sich zudem, wie vorteilhaft sich der Einsatz von IT-Systemen in den meisten Branchen auszahlt. Heimarbeit wurde von vielen Unternehmen gefördert, Universitäten konnten von zu Hause aus lehren und auch Teamtraining von Profisportlern wurde durch Einsatz von Videotelefonie ermöglicht. \newline
Dass die Verwendung von Informationssystemen und Internettechnologie nicht immer ohne Sicherheitsrisiken einhergeht, ist allerdings nicht zuletzt seit der "Corona-Krise" bekannt. Immer wieder gelingt es Angreifern an hochsensible Daten zu gelangen. Beispielsweise mussten Patienten eines tschechischen Krankenhauses verlegt werden, da die Uniklinik Brno gehackt wurde und somit der Betrieb der Klinik lahmgelegt wurde \cite{Holland2020}. Aber auch in der Industrie und bei Behörden kommt es immer öfter zu Angriffsversuchen.
Deshalb ist es für die jeweiligen IT-Abteilungen eine ständige Herausforderung die betriebseigenen Systeme sowohl proaktiv, als auch präventiv zu schützen.    \newline Um sich vor potentiellen Risiken zu schützen ist es schwierig das laufende System unter Stresstests zu setzen. Zu groß ist die Gefahr einen Ausfall herbeizuführen und den Betrieb dadurch zu unterbrechen. Als eine hilfreiche Methode hat sich in den letzten Jahren das Simulieren eines sogenannten Digitalen Zwillings herauskristallisiert. Dieser Digital Twin soll es Entwicklern ermöglichen ein reales System in einer geschützten Simulationsumgebung zu spiegeln und dieses unter beliebigen Bedingungen zu testen. \newline Eine weitere Herausforderung stellt die Präsentation von möglichen Angriffsvektoren dar. Die generierten Daten sind meistens nur schwer zu entziffern und daher nicht für den alltäglichen Umgang mit Bedrohungen im eigenen Netzwerk zu gebrauchen. Um dieses Problem zu lösen bietet es sich an einen Standard zu verwenden, der von vielen Aktoren genutzt wird. Ein derartiger Standard ist STIX (Structured Threat Information eXpression). \newline Im Rahmen eines Praxisseminars mit dem Titel "Angriffssimulation und strukturierte Datenerfassung" sollte ein industrielles Setting abgebildet werden und der Netzwerkverkehr eines simulierten Angriffs mitgeschnitten werden. Die erfassten Daten sollten, dann mittels STIX 2.x strukturiert für Cyber Threat Intelligence aufgearbeitet werden. 




%%%%%%%%%%%%%%%%%%%%%%%%%%%%%%%%%%%%%%%%%%%%%%%%%%%%%%%%%%%%%%%%%%%%%%%%%%%%%%


%%%%%%%%%%%%%%%%%%%%%%%%%%%%%%% Hauptteil %%%%%%%%%%%%%%%%%%%%%%%%%%%%%%%%

%%%%%%%%%%%%%%%%%%%%%%%%%%%%%%%%% Hauptteil %%%%%%%%%%%%%%%%%%%%%%%%%%%%%%%%%%

\chapter{Hauptteil}
\label{chap:Hauptteil}


\section{Beispiele}
Dies  ist eine Referenz auf ein Paper. Die Verwaltung der Referenzen erfolgt in der Datei References.bib. Zur Bearbeitung der Referenzen kann beispielsweise das Programm JabRef\protect{\footnote{\url{http://jabref.sourceforge.net/}}} verwendet werden.

Besonders interessant ist auch die automatische Erstellung des Abkürzungsverzeichnisses. Zuerst wird die Abkürzung definiert um bei erstmaliger Verwendung im Abkürzungsverzeichnis zu erscheinen: \ac{Bsp.}, \ac{SaaS}

Referenzen auf Grafiken: \ref{fig:Fig1}, \ref{img:subFig2}, \ref{img:subFigs}

%\begin{figure}
%  \centering
%  \includegraphics[width=0.95\textwidth]{Fig1.pdf}
%  \caption{Beispielgrafik}
%  \label{fig:Fig1}
%\end{figure}


%\begin{figure}
%  \centering
%  \subfigure[subfigure 1 \label{img:subFig1}]{\fbox{\includegraphics[width=0.45\textwidth]{Fig.png}}}\hfill
%  \subfigure[subfigure 2\label{img:subFig2}]{\fbox{\includegraphics[width=0.45\textwidth]{Fig2.pdf}}}\hfill
%  \caption{Beispiel subfigure}
%  \label{img:subFigs}
%\end{figure}

\begin{landscape}
Querseite, Kopf- und Fußzeile aber korrekt für gebundene Arbeit.
\end{landscape}

%\lstset{language=JAVA, breaklines=true, tabsize=2}
%\lstinputlisting[caption=HelloWorld,
%label=lst:HelloWorld]{listings/HelloWorld.java}


%%%%%%%%%%%%%%%%%%%%%%%%%%%%%%%%%%%%%%%%%%%%%%%%%%%%%%%%%%%%%%%%%%%%%%%%%%%%%%


%%%%%%%%%%%%%%%%%%%%%%%%%%%%%%%% Schluss %%%%%%%%%%%%%%%%%%%%%%%%%%%%%%%%%

%%%%%%%%%%%%%%%%%%%%%%%%%%%%%%%%%% Schluss %%%%%%%%%%%%%%%%%%%%%%%%%%%%%%%%%%%

\chapter{Schluss}
\label{chap:Schluss}

\blindtext

\blindtext

%%%%%%%%%%%%%%%%%%%%%%%%%%%%%%%%%%%%%%%%%%%%%%%%%%%%%%%%%%%%%%%%%%%%%%%%%%%%%%


%%%%%%%%%%%%%%%%%%%%%%%%%%%%%%%%%%%%%%%%%%%%%%%%%%%%%%%%%%%%%%%%%%%%%%%%%%



%%%%%%%%%%%%%%%%%%%%%%%%% Literaturverzeichnis %%%%%%%%%%%%%%%%%%%%%%%%%%%

\newpage

\phantomsection
\addcontentsline{toc}{chapter}{Appendices}
\appendix
\appendixpage

%%\appendix
%%\phantomsection
%\renewcommand*\appendixpagename{Anhang} 
%%\appendixpage
%%\addappheadtotoc

\chapter{Erster Anhang}
\label{chap:anhang_1}

\chapter{Zweiter Anhang}
\label{chap:anhang_2}

\section{Anhang}
\label{sec:anhang_2_1}

\section{Anhang}
\label{sec:anhang_2_2}

\clearpage

%%%%%%%%%%%%%%%%%%%%%%%%% Literaturverzeichnis %%%%%%%%%%%%%%%%%%%%%%%%%%%
\phantomsection
\addcontentsline{toc}{chapter}{Literaturverzeichnis}
\bibliographystyle{alphadin}
\bibliography{References}

\clearpage

%
%%%%%%%%%%%%%%%%%%%%%%%%%%%%%%%%%%%%%%%%%%%%%%%%%%%%%%%%%%%%%%%%%%%%%%%%%%
%% Eidesstattliche Erklärung %%%%
%
% nur bei Abschlussarbeiten!
%

\thispagestyle{empty}
\phantomsection
\label{erklaerung}
%\addcontentsline{toc}{chapter}{Erklärung}
\pdfbookmark[0]{Eidesstattliche Erklärung}{erklaerung}

\setlength{\parindent}{0em}

\markboth{\uppercase{Eidesstattliche Erklärung}}{}
\textbf{\large{Erklärung an Eides statt}}

\vspace*{20pt}
Hiermit versichere ich an Eides statt, dass ich die vorliegende Arbeit selbständig verfasst und keine anderen als die angegebenen Quellen und Hilfsmittel benutzt habe. Die aus fremden Quellen direkt oder indirekt übernommenen Gedanken sind als solche kenntlich gemacht. Die Arbeit wurde bisher in gleicher oder ähnlicher Form keiner anderen Prüfungsbehörde vorgelegt und auch nicht veröffentlicht.

Die elektronische Ausfertigung der Arbeit habe ich bereits beim Prüfer eingereicht.

\vspace*{65pt}


Regensburg, den \abgabedatum

\vspace*{60pt}


\authorname

Matrikelnummer \matrikelnr

\end{document}
