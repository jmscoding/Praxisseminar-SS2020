%%%%%%%%%%%%%%%%%%%%%%%%%%%%%%%%% Hauptteil %%%%%%%%%%%%%%%%%%%%%%%%%%%%%%%%%%

\chapter{Hauptteil}
\label{chap:Hauptteil}


\section{Beispiele}
Dies  ist eine Referenz auf ein Paper. Die Verwaltung der Referenzen erfolgt in der Datei References.bib. Zur Bearbeitung der Referenzen kann beispielsweise das Programm JabRef\protect{\footnote{\url{http://jabref.sourceforge.net/}}} verwendet werden.

Besonders interessant ist auch die automatische Erstellung des Abkürzungsverzeichnisses. Zuerst wird die Abkürzung definiert um bei erstmaliger Verwendung im Abkürzungsverzeichnis zu erscheinen: \ac{Bsp.}, \ac{SaaS}

Referenzen auf Grafiken: \ref{fig:Fig1}, \ref{img:subFig2}, \ref{img:subFigs}

%\begin{figure}
%  \centering
%  \includegraphics[width=0.95\textwidth]{Fig1.pdf}
%  \caption{Beispielgrafik}
%  \label{fig:Fig1}
%\end{figure}


%\begin{figure}
%  \centering
%  \subfigure[subfigure 1 \label{img:subFig1}]{\fbox{\includegraphics[width=0.45\textwidth]{Fig.png}}}\hfill
%  \subfigure[subfigure 2\label{img:subFig2}]{\fbox{\includegraphics[width=0.45\textwidth]{Fig2.pdf}}}\hfill
%  \caption{Beispiel subfigure}
%  \label{img:subFigs}
%\end{figure}

\begin{landscape}
Querseite, Kopf- und Fußzeile aber korrekt für gebundene Arbeit.
\end{landscape}

%\lstset{language=JAVA, breaklines=true, tabsize=2}
%\lstinputlisting[caption=HelloWorld,
%label=lst:HelloWorld]{listings/HelloWorld.java}


%%%%%%%%%%%%%%%%%%%%%%%%%%%%%%%%%%%%%%%%%%%%%%%%%%%%%%%%%%%%%%%%%%%%%%%%%%%%%%
