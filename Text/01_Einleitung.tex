%%%%%%%%%%%%%%%%%%%%%%%%%%%%%%%%% Einleitung %%%%%%%%%%%%%%%%%%%%%%%%%%%%%%%%%

\chapter{Einleitung}
\label{chap:Einleitung}


Wie sich in den Anfangsmonaten des Jahres bereits gezeigt hat, ist unser Wirtschaftssystem bei Weitem fragiler, als es viele Experten vorhersagen konnten. Innerhalb kürzester Zeit versetzte ein kleiner Virus die komplette Weltwirtschaft in einen ungewollten Ruhemodus.
Es zeigte sich zudem, wie vorteilhaft sich der Einsatz von IT-Systemen in den meisten Branchen auszahlt. Heimarbeit wurde von vielen Unternehmen gefördert, Universitäten konnten von zu Hause aus lehren und auch Teamtraining von Profisportlern wurde durch Einsatz von Videotelefonie ermöglicht. \newline
Dass die Verwendung von Informationssystemen und Internettechnologie nicht immer ohne Sicherheitsrisiken einhergeht, ist allerdings nicht zuletzt seit der "Corona-Krise" bekannt. Immer wieder gelingt es Angreifern an hochsensible Daten zu gelangen. Beispielsweise mussten Patienten eines tschechischen Krankenhauses verlegt werden, da die Uniklinik Brno gehackt wurde und somit der Betrieb der Klinik lahmgelegt wurde \cite{Holland2020}. Aber auch in der Industrie und bei Behörden kommt es immer öfter zu Angriffsversuchen.
Deshalb ist es für die jeweiligen IT-Abteilungen eine ständige Herausforderung die betriebseigenen Systeme sowohl proaktiv, als auch präventiv zu schützen.    \newline Um sich vor potentiellen Risiken zu schützen ist es schwierig das laufende System unter Stresstests zu setzen. Zu groß ist die Gefahr einen Ausfall herbeizuführen und den Betrieb dadurch zu unterbrechen. Als eine hilfreiche Methode hat sich in den letzten Jahren das Simulieren eines sogenannten Digitalen Zwillings herauskristallisiert. Dieser Digital Twin soll es Entwicklern ermöglichen ein reales System in einer geschützten Simulationsumgebung zu spiegeln und dieses unter beliebigen Bedingungen zu testen. \newline Eine weitere Herausforderung stellt die Präsentation von möglichen Angriffsvektoren dar. Die generierten Daten sind meistens nur schwer zu entziffern und daher nicht für den alltäglichen Umgang mit Bedrohungen im eigenen Netzwerk zu gebrauchen. Um dieses Problem zu lösen bietet es sich an einen Standard zu verwenden, der von vielen Aktoren genutzt wird. Ein derartiger Standard ist STIX (Structured Threat Information eXpression). \newline Im Rahmen eines Praxisseminars mit dem Titel "Angriffssimulation und strukturierte Datenerfassung" sollte ein industrielles Setting abgebildet werden und der Netzwerkverkehr eines simulierten Angriffs mitgeschnitten werden. Die erfassten Daten sollten, dann mittels STIX 2.x strukturiert für Cyber Threat Intelligence aufgearbeitet werden. 




%%%%%%%%%%%%%%%%%%%%%%%%%%%%%%%%%%%%%%%%%%%%%%%%%%%%%%%%%%%%%%%%%%%%%%%%%%%%%%
